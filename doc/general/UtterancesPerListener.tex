\documentclass[12pt]{article}

%\usepackage{natbib}
\usepackage{xspace}
%\usepackage{epsfig}
%\usepackage{graphics}
\usepackage{color}
\usepackage{relsize} % provides \relscale, \larger, etc. to be used *in the document*
%\usepackage{hyperref}\hypersetup{pdftex,colorlinks=true,allcolors=blue}

%\usepackage{anyfontsize} % e.g.: \fontsize{13}{14.5}\selectfont cmr12
                         % [I think this is 13 on 14.5, 
			 % i.e. 13pt with 1.5pts extra space between lines.]

%\usepackage{amsmath}

%\newcommand{\SetFont}[1]{\setmainfont[Mapping=tex-text]{#1}} % font sel abbrev
\newcommand{\SetFont}[1]{\setmainfont[Ligatures=TeX]{#1}} % preferred to "Mapping=tex-text"
%\newcommand{\SetFont}[1]{\setmainfont[OpticalSize=0]{#1}} % font sel abbrev
%\newcommand{\SetFont}[1]{\setmainfont[Mapping=tex-text,Ligatures=TeX,Scale=1.20]{#1}} % font sel abbrev

%\setmainfont[Mapping=tex-text]{GaramondNo8}

% Args: EB Garamond/GaramondNo8/Garamond/Crimson
\newcommand{\UseGaramond}[1]{%
	\usepackage[urw-garamond]{mathdesign}  % goes before fontenc, fontspec, etc.
	\usepackage[T1]{fontenc}
	\usepackage{fontspec,xltxtra,xunicode}
	\SetFont{#1}
}

% Args: TeX Gyre Pagella/URWPalladioL/Palatino/Palatino Linotype/Book Antiqua
% [broken?]
\newcommand{\UsePalatino}[1]{%
 	\usepackage[T1]{fontenc}
 	\usepackage{fontspec,xltxtra,xunicode}
 	%\usepackage{tgpagella} % better to select via Mac fonts:
 	\SetFont{#1}
 	\usepackage{unicode-math} 
 	%\setmathfont{texgyrepagella-math.otf}
 	\setmathfont{TeX Gyre Pagella Math}
}

%\UseGaramond{GaramondNo8}
\UseGaramond{EB Garamond}
%\UsePalatino{TeX Gyre Pagella}
%\UsePalatino{Book Antiqua}


%\renewcommand{\baselinestretch}{1.66}\normalsize % my trad double-space
%\renewcommand{\baselinestretch}{1.25}\normalsize % 12 on 15 (if base is 12)
%\renewcommand{\baselinestretch}{1.17}\normalsize % 12 on 14 (if base is 12)
\renewcommand{\baselinestretch}{1.08}\normalsize % 12 on 13 (if base is 12)
%\renewcommand{\baselinestretch}{1.10}\normalsize % 

\usepackage{vmargin}
\setpapersize{USletter}
%\setmarginsrb{1in}{1.15in}{1in}{1.15in}{0pt}{0in}{0pt}{.3in} % MBP
%\setmarginsrb{1in}{1.40in}{1in}{1.40in}{0pt}{0in}{0pt}{.3in} % MBA
%\setmarginsrb{1.25in}{1in}{1.25in}{1in}{0pt}{0in}{0pt}{.3in}
\setmarginsrb{1.0in}{1in}{1.0in}{1in}{0pt}{0in}{0pt}{.3in}

\setlength{\parindent}{0in}  % Don't indent paragraphs
\setlength{\parskip}{2.5ex}    % Add space between paragraphs

\newcommand{\ie}{i.e.\@\xspace}
\newcommand{\eg}{e.g.\@\xspace}
\newcommand{\cf}{cf.\@\xspace}
\newcommand{\etc}{etc.\@\xspace}
\newcommand{\viz}{viz.\@\xspace}
\newcommand{\vs}{vs.\@\xspace}

\newcommand{\pr}{\mathsf{P}} % probability
\newcommand{\expct}{\mathsf E\,}
\newcommand{\var}{\sigma^2}
%\newcommand{\var}{\mathop{\mathrm{var}}\nolimits}
\newcommand{\stdv}{\sigma}
%\newcommand{\stdv}{\mbox{stddev}}
%\newcommand{\comb}[2]{\left(^{#1}_{#2}\right)}
%\newcommand{\comb}[2]{\left(^{#1}_{#2}\rule{0em}{1em}\right)}
%\newcommand{\comb}[2]{\binom{#1}{#2}} % from amsmath
\newcommand{\comb}[2]{{{#1}\choose{#2}}}
% [0-width rule functions as a strut, makes parentheses a good size.]

\newcommand{\cmnt}[1]{}
%\renewcommand{\cmnt}[1]{{\color{red}[#1]}}

\newcommand{\fn}[1]{\footnote{#1}}

\begin{document}
\relscale{1.3} % change size globally

{\Large\sc How many utterances does each listener receive?}

Let there be $N$ popco persons all in the same fully
interconnected group.  (For the moment, ignore pundits.)  Assume
that each person has {\sf max-talk-to} = $m$.

Then on any given tick, there are $N$ speakers with up to $m$ utterances
each, \ie up to $mN$ utterances in total, distributed between $N$
listeners.  The probability that any given listener $l$ will
receive a particular utterance is $p=\frac{1}{N}$.

How many utterances does each listener receive?

The probability that a given listener $l$ will receive $k$
utterances during a tick follows a binomial distribution:
\[
    \pr(X=k) = \comb{mN}{k} \left(\frac{1}{N}\right)^k
    \left(1-\frac{1}{N}\right)^{mN-k} \;.
\]
The expected number of utterances received by a given listener is
\[
    \expct(X) = \sum_{k=0}^{mN}k\comb{mN}{k}
    \left(\frac{1}{N}\right)^k \left(1-\frac{1}{N}\right)^{mN-k}
    = mNp = mN\frac{1}{N} = m \;,
\]
and the variance is
\[
    \var(X) = mNp(1-p) = mN\frac{1}{N}\left(1-\frac{1}{N}\right)
    = m\left(\frac{N-1}{N}\right) \;.
\]
The standard deviation is therefore
\[
    \stdv(X) = \sqrt{m}\sqrt{\frac{N-1}{N}} \;.
\]
For a slightly large $N$, the second term is close to 1.  For
example, the NetLogo Bali model has $N=172$ subaks, so if {\sf
max-talk-to} $=m=4$, and all subaks generate that the max number
ofutterances, the mean number of utterances per listener will be
4, and the standard deviation will be slightly less than $2$.

(All correct?)

\vspace{4ex}


{\Large\sc What fraction of utterances are selected by success bias?}

Similarly, since the number of utterances produced by any one
speaker is $m$, the expectation and variance of the number of utterances
received by a given listener $l$ from any one speaker $s$, are
\[
\expct(X_s) = mp = \frac{m}{N}
\]
%
\[
\var(X_s) = m\frac{1}{N}\left(1 - \frac{1}{N}\right) = \frac{m(N-1)}{N^2}
\]
and the standard deviation is
\[
\stdv(X_s) = \frac{\sqrt{m(N-1)}}{N} \;.
\]

If we assume that a listener's success-bias function always
selects a single speaker's utterances on any given tick (which
would generally obtain if success varies continuously and is
rarely equal to its maximum value), then we might we treat the
ratio between the above two means as an estimate of the average
fraction of most-sucessful utterance among all utterances
received by a listener:
\[
    \frac{\frac{m}{N}}{\;\;\;m\;\;\;} = \frac{m^2}{N}
\]
For example, for $N=100$ persons, with $m=5$ utterances per person,
roughly 1/4 of the utterances received, on average, by a
listener, would be the most successful for that listener on that
tick.

(Yeah, but that's not the real way to calculate this.  A ratio of
expectations isn't an expectation, in general.)

\vspace{3ex}

So let's try again.

The probability that $l$ gets $j$ utterances from speaker $s$,
where $j\leq m$, is:
\[
    \pr(X_s=j) = \comb{m}{j} \left(\frac{1}{N}\right)^j
    \left(1-\frac{1}{N}\right)^{m-j} \;.
\]
($j$ must also be $\leq k$.)

The probability that a given listener $l$ will receive $k-j$
utterances from any person but $s$ during a tick is:
\[
    \pr(X=(k-j)) = \comb{m(N-1)}{k-j}
    \left(\frac{1}{N-1}\right)^{k-j}
    \left(1-\frac{1}{N-1}\right)^{m(N-1)-(k-j)} \;.
\]

The probability that $l$ receives $k$ utterances, and $j$ of
them are from $s$, is the probability that $l$ receives $j$
utterances from $s$ and $k-j$ utterances from all of the other
speakers is 
\[
\pr(X_s=j) \; \pr(X=(k-j)) 
\]
since each person generates utterances independently of every
other. (Is this correct?)


\end{document}
